\chapter{Introduction and Least Squares}
For this lecture, we will traverse through the Spring 2023 versions of course logistics, and then an introduction towards optimization.

\section{An Introductory Prompt}
In this course, we discuss a problem in Computer Science called, ``optimization''.
Understanding optimization is not limited to just understanding mathematical details, and has more to do with ``problem formulation'': for whatever technology comes, optimization is an involved technique of critical thinking.

\section{Administratives, Summary}
\textbf{Logistics.} For Discussion, Homework policies, please check the lecture slides. Friday sections are equivalent to the subsequent Monday sections. \\
\textbf{Advices from Instructor.} Do homeworks, collaborate with people. \\
\textbf{New: Projects.} We will have an option to do a project at the end of the semester, where we complete a project of choice.
This is an effort to bridge the gap between students and research experience; where, provided a research paper, the student will extend that literature at the end of project.
This will count towards grade and remove weight from exams if the project grade helps.
Schemes will be announced. Projects will be in groups, released after the midterm and due during RRR week.

\section{Introduction to The Subject Topic: Optimization}
Optimization is an approach to problems. \\
It is the technique where we attempt to optimize some statistic resembling of a result.
For example, in machine learning from DATA C100 (or EECS 16A/B), we have chosen appropriate loss functions for our learning task to characterize the performance of a model:
\begin{bindenum}
    \item Minimizing the MSE of a linear regresssion model.
    \item Minimizing the cross-entropy loss of a logistic regression model.
    \item Minimizing the distance traveled by an agent in a maze.
\end{bindenum}
Here, we model learning tasks and attempt to optimize a metric. Depending on how we model and represent a problem, the model will perform differently on its learning task.
Therefore, optimization is a study about ``picking the right loss function'' for some same objective.
We consider these design choices, study how they affect learning process.

For another example, Air Traffic control is another optimization problem (at the point of writing, US has just experienced a flight paralysis); for another example, queueing and revenue computation are also common optimization problems.
Optimization itself is omnipresent in modern applications.

In this course, we will learn about:
\begin{bindenum}
    \item Low rank approximation
    \item Ridge regression
    \item Stochastic Gradient Descent
    \item Dual Program (which always provides a lower bound for some optimization problem)
    \item Applications: LQR Control, Classification, SVM
\end{bindenum}

Let's get into the Math now!!!

\subsection{An Example of Problem Formulation}
Say, we work in an oil production firm (I know, very US).
We are allowed to make 10,000 barrels of crude oil, which may be produced into either \textit{Jet Fuel} or \textit{Gasoline}. \\
For every barrel of Jet Fuel produced, a revenue of 30 cents; for Gasoline, 20 cents. Meanwhile, 1 barrel of crude oil produces 0.6 barrels of Jet Fuel or 0.7 barrels of Gasoline. \\
Furthermore, the firm demands that we produce more than some amount of Jet Fuel and Gasoline (respectively, say, 1000 and 2000 barrels). \\
Finally, there are transportation capacities, where 180000 barrel miles are available. Distributing Gasoline costs 30 miles, and distributing Jet Fuel costs 10 miles. \\

We have now been presented a prompt: provided the above conditions, how can I maximize my revenue? \\
Let us first formulate this prompt mathematically:
\[
    \begin{cases}
        x_j = \text{ Quantity of Jet Fuel in barrels} \\
        x_g = \text{ Quantity of Gasoline in barrels}
    \end{cases}
\]
The revenue we generate would be formulated as
\[R(x_j, x_g) = 0.3 x_j + 0.2 x_g\]
such that we are presented the \textbf{constraints}:
\[
    \begin{cases}
        \text{Production Quantity Minimum: } &x_j \geq 1000, x_g \geq 2000 \\
        \text{Total Available Resources: } &\frac{x_j}{0.7} + \frac{x_g}{0.6} \leq 10000 \\
        \text{Transportation Capacity: } &10 x_j + 30 x_g \leq 180000
    \end{cases}
\]
which are mathematically translated from the above English text.

In an optimization framework, we formualte a problem with the following frame:
\begin{ln-explain}{Formulating Optimization Problems}{}
    In a general optimization problem, we attempt to minimize some function $f_0(\vec{x})$, such that some constraint exists:
    \[\forall i \in \{1, \dots, m\}, f_i(\vec{x}) \leq b_i\]
    and here, $\vec{x}$ is listed as a representation of some system's state, existing within the domain of possible inputs (states). \\
    The general objective of optimization is to find a state of system that minimizes $f_0(\vec{x})$, which we mathematically express the solution as:
    \[\vec{x}^* = {argmin}_{\forall i \in \{1, \dots, m\}, f_i(\vec{x}) \leq b_i} f_0(\vec{x})\]
\end{ln-explain}
See the example from above, and try matching the parts of optimization problem with the above framework!

There are some more types of optimization problems! One of them is the famous Least Squares Regression:

\section{Least Squares Regression}
The problem statement:
\begin{center}
    For some matrix $A$ and a vector $\vec{b}$, solve for:
    \[\min_{\vec{x}} {\lvert\lvert A \vec{x} - \vec{b} \rvert\rvert}_2^2\]
\end{center}

Such problem can be widely applied to many mathematical problems, such as regression and projection.

\begin{ln-explain}{Formulating Least Squaresd Regression}{}
    Provided a dataset of points:
    \[\{(x_1, y_1), \dots, (x_n, y_n)\}\]
    where we attempt to model a linear equation
    \[y = mx + c\]
    
    First of all, what is the variable we attempt to search/optimize for? It would be $m, c$ that are the unknowns. \\
    To formulate the Least Squares Problem, we would need to formulate our linear equation for points into some matrix-vector multiplication, and knowing that we are solving for $m$ and $c$, the formulation follows:
    \[
        A
        \begin{bmatrix} m \\ c \end{bmatrix}
        = \vec{b}
    \]
    Let us now fill in the contents of $A$ and $\vec{b}$ for formulation, where $A$ and $\vec{b}$ are of known quantities:
    \[
        \begin{bmatrix} \vec{x} & \vec{1} \end{bmatrix}
        \begin{bmatrix} m \\ c \end{bmatrix}
        = \vec{y}
    \]
\end{ln-explain}
From prior coursework we also know that there is a closed-form solution:
\[\vec{x}^* = {(A^T A)}^{-1} A^T \vec{b}\]
for matrices $A$ with nontrivial nullspaces (see EECS16A for a proof that $N(A) = N(A^T A)$).

\begin{ln-explain}{Closed-Form Solution of Least Squares}{}
    To minimize the L2 norm of difference between $A \vec{x}$ and $\vec{b}$, we attempt to find a vector $\vec{x}$ that minimizes the differences between them. \\
    Note that for any choice of $\vec{x}$, it is by definition that $A \vec{x} \in Col(A)$. \\

    We now face two claims that we resolve to proceed on solving this problem:
    \begin{bindenum}
        \item[1.] The endpoint of ${proj}_{Col(A)} \vec{b}$ is the point closest to the endpoint of $\vec{b}$ (if they share a same starting point) in $Col(A)$.
        \item[2.] $\vec{OP} = A \vec{x}^*$, where $\vec{x}^*$ has the closed-form solution as previously described.
    \end{bindenum}

    Claim 1 can be proven by contradiction: assume another point $C$ closer than $P$ to $B$ (such that $\vec{OB}$ is closer to $\vec{b}$). However, Pythagorean Theorem argues against it. \\
    Claim 2 is a series of algebraic manipulation as outlined in previous courseworks, based on that the error vector $\vec{e} = \vec{b} - A\vec{x}$ should be orthogonal to the columnspace $Col(A)$ by geometry:
    \begin{align*}
        A^T \vec{e} &= 0 \\
        A^T (A \vec{x}^* - \vec{b}) &= 0 \\
        A^T A \vec{x}^* &= A^T \vec{b} \\
        \vec{x}^* &= {(A^T A)}^{-1} A^T \vec{b}
    \end{align*}
    Such result is also known as the \textbf{Normal Equation}.
\end{ln-explain}

Interestingly, Least Squares Algorithm has a quadratic property, causing it to be some parabolic object that performs a \textbf{convex function} to optimize along; that is, the local minimum of this function is a global minimum.
