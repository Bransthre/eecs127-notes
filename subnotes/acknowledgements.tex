\section{Preface}
\textit{This section is directly copy-pasted from my \href{https://github.com/Bransthre/eecs127-notes}{GitHub repo} for the note.}

eecs127 moment. \\
Hi I'm Brandon. You might know me from work.
If not, I think you should be glad you don't, or you would have had to deal with my EECS 16A Puns. \\
I update this note after every lecture and before exams (which I will notify about if I run a revision, or you can watch the repository). \\
Since I make these lecture notes by transcribing lecture contents on the fly, you would probably \underline{expect immature pedagogical formatting} (which immediately appears) and typos in these efforts.

\subsection{First of all, My Notes are Not Substitutes to the Course Readers}
My notes are my own transcriptions for EECS 127 lecture notes.
On a professional note, yes, you may use these for personal purposes and gaining extra insights on in-course contexts, but this is by no means a pedagogical substitute of the original course notes, even if it describes all the concepts of the course reader.

On a professional note, once again, the official course readers are structured in a pedagogical imperative: it connects concepts to optimize student understanding and the coherency of course content (even if it might have been confusing on the first reads).
This course note does not. I was not obligated to connect concepts. I was only writing the work to organize information for my own sake.
This work is not specifically written for the public just like my CSM resources were. My notes are by no means a substitute of EECS 127 readers.

But, thinking from the other angle, we may consider these notes I produced as a complementary resource for EECS 127.

\subsection{How did Brandon Use This Note?}
Great Question. I'm glad you asked. Cyrus.

I used this note extensively to lookup summaries of concepts when writing homeworks and mock exams, as well as to force myself to read through derivations on lecture notes via transcribing them onto LaTeX and running a rigorous revision later.
I have also shared my notes to others (as you see, it is hosted as an open source project on Github), in the hope that they can use these notes as a lookup for summaries.

You can expect to see more formal course contents starting at the later half of Note 1.

\subsection{What is the Last Four Digits of Your Social Security Number?}
gXcQ
