\chapter{Interior Point Methods}

\section{Introduction to Interior Point Method}
When working with unconstrained QPs and linear equality constrained QPs, we have generally convenient formulas to solve such problems.
For any program that is not a QP but can be phrased as one, we may approximate it in a second-order Taylor polynomial under unconstrained settings.
Meanwhile, equality constrained Newton's Method was developed in prior lectures via the help of KKT Conditions and matrix inversion. \\
Well, what about solving other constrained programs?

Interior Point Method is a powerful tool that we can use to solve constrained optimization problems, specifically on programs with linear equality and linear inequality constraints. \\

\section{Barrier Function}
Suppose that we face an optimization problem:
\[
    \min_{
        \substack{
            A \vec{x} = \vec{b} \\
            \forall i, f_i(\vec{x}) \leq 0
        }
    } f_o (\vec{x})
\]
We may add an indicator function to our optimization problem, where the indicator function outputs infinity for any point that does not satisfy the inequality constraint, and $0$ for any point that satisfies the inequality constraint. \\
Then, we may rewrite the problem as:
\[
    \min_{
        A \vec{x} = \vec{b}
    } f_o (\vec{x}) + \sum_i \mathbbm{1}_{f_i(\vec{x}) \leq 0}
\]
Such indicator function is known as the barrier function.
\begin{ln-define}{Barrier Function}{}
    In constrained optimization, a field of mathematics, a barrier function is a continuous function whose value on a point increases to infinity as the point approaches the boundary of the feasible region of an optimization problem
    (\href{https://en.wikipedia.org/wiki/Barrier_function}{Wikipedia}).
\end{ln-define}
Suppose rather than using a piecewise indicator function, we work with an approximated one, such that:
\[
    I_{\rm approx} (u) = -\frac{1}{t} \log(-u)
\]
is the logarithmic barrier function. \\
Then, our new optimization problem becomes
\begin{align*}
    p^* &= \min_{
        A \vec{x} = \vec{b}
    } f_o (\vec{x}) + \frac{1}{t} \Phi(\vec{x}) \\
    \Phi(\vec{x}) &= \sum_{i = 1}^n -\log(-f_i(\vec{x}))
\end{align*}
