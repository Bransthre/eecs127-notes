\chapter{Second Order Cone Programs}

\section{Second Order Cone Programs (SOCP)}
Let us first define a ``cone'':
\begin{ln-define}{Cone}{}
    A set $\mathcal{K}$ is a cone if:
    \[
        \forall \alpha \geq 0, (\vec{x} \in \mathcal{K} \implies \alpha \vec{x} \in \mathcal{K})
    \]
    For example,
    \[
        \{ (x_1, x_2) \big| |x_1| \leq x_2 \}, \{ (x_1, x_2) \big| x_2 \geq 0 \}
    \]
\end{ln-define}
Furthermore, we may define a convex cone as a cone that is also convex:
\begin{ln-define}{Convex Cone}{}
    \begin{quote}
        Cone but convex. -- Prof. Ranade, 10:40AM April 6th 2023 at Lewis 100
    \end{quote}
\end{ln-define}
Next, we may define a polyhedral cone as follows:
\begin{ln-define}{Polyhedral Cone}{}
    Suppose we have a polyhedron $\mathcal{P} = \{ \vec{x} \big| A \vec{x} = \vec{b}\}$. \\
    Then, a polyhedral cone is the set:
    \[
        \mathcal{K} = \{
            (\vec{x}, t) \big| A \vec{x} \leq \vec{b} t, t \in \R, t \geq 0
        \}
    \]
\end{ln-define}
And, a second order cone may thus be defined as:
\[
    \mathcal{K} = \{(\vec{x}, t) \big| \pnorm{\vec{x}}{2} \leq t \} \subset \R^{|\vec{x}| + 1}
\]

Now, let us consider this general class of problem, called the SOCP:
\[
    \min_{
        \forall i \in \{1, \dots, m\}, \pnorm{A_i \vec{x} + b_i}{2} \leq \vec{c}_i^T \vec{x} + d_i
    } \vec{g}^T \vec{x}
\]
Why is it called an SOCP, instead of something like, I don't know, Among Us? \\
Let us considerthe variables:
\[
    \begin{cases}
        \vec{y} = A_i \vec{x} + b_i \\
        t = \vec{c}_i^T \vec{x} + d_i
    \end{cases}
\]
then we see that SOCP problems have a constraint equivalent to the set of points $(\vec{y}, t)$ that is being contained by some second order cone:
\[
    \mathcal{K} = \{\vec{x} \big| \pnorm{A_i \vec{x} + b_i}{2} \leq \vec{c}_i^T \vec{x} + d_i \}
\]
with traingle inequality we may even prove that the feasible set of this problem, this set is convex.
You can't imagine how long I've waited to drop this line, but,
\begin{quote}
    This is left as an exercise to the reader.
\end{quote}

\subsection{Facility Location Problem}
Suppose we have a few asylums spread across the country (eg., University of California, Berkeley; University of Currying, Brocolli; University of Critish Bolumbia...), how may we allocate resources across these centers to minimize the amount of commute in distributing resources? \\
We may formulate this problem as:
\[
    \min_{\vec{x}} \frac{1}{m} \sum_{i = 1}^m \pnorm{A_i \vec{x} - b_i}{2}
\]
and the argument along which we may debate this is a SOCP is that we may first state,
\[
    z_i = \pnorm{A_i \vec{x} - b_i}{2}
\]
such that we now have a linear objective function, and the problem is now rephrased as:
\[
    \min_{
        \substack{
            x \\
            \pnorm{A_i \vec{x} - b_i}{2} \leq z_i
        }
    } \sum_{i = 1}^n z_i
\]
Furthermore, we have thus developed a constraint in the form of $\pnorm{A_i \vec{x} + b_i}{2} = \vec{c}_i^T \vec{x} + d_i$.
We may then relax the constraint to obtain inequality constraints instead, with optimum preserved.

Furthermore, we may perform customizations to maximize specific utilities, where, for example, the asylum with least resources available (my university in this case?) would need the better treatment from this problem. \\
Through this expression of ``min-max utility'', upon which we rephrase the problem as:
\[
    \min_x \max_i \pnorm{\vec{x}_i - \vec{y}_i}{2}
\]
we may then rephrase this as a SOCP problem. \\
Once again, the context of the problem is to minimize $\vec{x}$ on the maximum possible distance (the worst-off asylum's situation of receiving no funding for education). \\
Now, suppose that $\max_i \pnorm{\vec{x}_i - \vec{y}_i}{2} = R$. Then, we may simplify the problem into
\begin{align*}
    \min_{
        \substack{
            x, R \\
            \forall i, \pnorm{A_i \vec{x} - b_i}{2} \leq R
        }
    } R
\end{align*}
which is a SOCP problem.
