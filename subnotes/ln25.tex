\chapter{Optimization Applications I: LQR (Linear-Quadratic Regulator) Control}

\section{Introduction to Model-Based Control}
Control problems have been approached on some model-based method; for example, we mathematically model the relationship of states and controls as:
\[
    \vec{x}_{t + 1} = A \vec{x}_t + B \vec{u}_t
\]
where $\vec{x}_t$ represents the state of a system at timestamp $t$.

We would like to assess the cost of an LQR to be:
\[
    \sum_{t = 0}^{N - 1} \frac{1}{2} (\vec{x}_t^T Q \vec{x}_t + \vec{u}_t^T R \vec{u}_t) + \frac{1}{2} \vec{x}_N^T Q_F \vec{x}_N
\]
where $Q, Q_F, R$ are symmetric and positive-semidefinite.
Here, the term $\vec{x}_t^T Q \vec{x}_t$ accounts for the penalty of a current state, and $\vec{u}_t^T R \vec{u}_t$ provides a cost on the control terms $u$. \\
We perform a minimization on this control, where we also pose the following constraints to phrase the minimization problem as:
\[
    \min_{
        \substack{
            \vec{u}_t \\
            \forall t = \{0, \dots, N - 1\}, \vec{x}_{t + 1} = A \vec{x}_t + B \vec{u}_t \\
            \vec{x}_0 = \vec{x}_{\rm init}
        }
    } \sum_{t = 0}^{N - 1} \frac{1}{2} (\vec{x}_t^T Q \vec{x}_t + \vec{u}_t^T R \vec{u}_t) + \frac{1}{2} \vec{x}_N^T Q_F \vec{x}_N
\]
For one of the methods, we may employ dynamic programming with back propagation of information. This is among one of the standard methods to resolve LQR Control problem. \\
On a simialr logic, as each control and state at a timestamp is dependent upon some term of a timestamp before it, we can formulate large optimization problem as we write every expression in terms of $\vec{x}_0$ and $\vec{u}$s.
However, this is computationally expensive. We must find a more elegant method: \textbf{A Duality-Based, Iterative Trick}.

\section{LQR Control Via Duality and Certificate}
Suppose that we rewrite the Lagrangian of our above objectiev function as:
\[
    \mathcal{L}(\vec{x}, \vec{u}, \vec{\nu}) =
    \sum_{t = 0}^{N - 1} \frac{1}{2} (\vec{x}_t^T Q \vec{x}_t + \vec{u}_t^T R \vec{u}_t) + \frac{1}{2} \vec{x}_N^T Q_F \vec{x}_N + \vec{\nu}_0^T (\vec{x}_{\rm init} - \vec{x}_0) + \sum_{t = 0}^{N - 1} \vec{\nu}_{t + 1}^T (A \vec{x}_t + B \vec{u}_t - \vec{x}_{t + 1})
\]

The associated KKT Conditions for such problem can be organized as follows. \\
First of all, let us derive the gradient of our Lagrangian w.r.t. its inputs at some specific timestamp $t$:
\begin{align*}
    \nabla_{\vec{u}_t} \mathcal{L} &=
    \begin{cases}
        Q \vec{x}_t + A^T \vec{\nu}_{t + 1} - \vec{\nu}_t &, 0 \leq t < N \\
        Q_F \vec{x}_N - \vec{\nu}_N &, t = N
    \end{cases} \\
    \nabla_{\vec{x}_t} \mathcal{L} &= R \vec{u}_t + B^T \vec{\nu}_{t + 1}
\end{align*}
Consequently, we acquire that,
\[
    \begin{cases}
        \vec{u}_t &= - R^{-1} B^T \vec{\nu}_{t + 1} \\
        \text{\rm Adjoint System:}
        &\begin{cases}
            \vec{\nu}_t = A^T \vec{\nu}_{t + 1} + Q \vec{x}_t \\
            \vec{\nu}_N = Q_F \vec{x}_N
        \end{cases}
    \end{cases}
\]

We now make a reasonable guess that there exists some linear relationship between $\vec{x}$ and $\vec{\nu}$.
\[
    \vec{\nu}_t = P_t \vec{x}_t, P_N = Q_F
\]
Then, using backwards induction, we would like to prove the guess:
\begin{ln-explain}{Proof of the Ricatti Equation}{}
    \textbf{Proof:} \\
    \textbf{Base Case.} We know that $\vec{\nu}_N = Q_F \vec{x}_N$. \\
    \textbf{Induction Hypothesis.} Suppose that for some timestamp t, $\vec{\nu}_t = P_t \vec{x}_t$. \\
    \textbf{Inductioon Step.} Then, let us substitute the equations we found above for $\vec{\nu}$ and $\vec{x}$ into the above expression:
    \begin{align*}
        \vec{\nu}_t &= P_t \vec{x}_t \\
        &= P_t (A \vec{x}_{t - 1} + B \vec{u}_{t - 1}) \\
        &= P_t (A \vec{x}_{t - 1} - B R^{-1} B^T \vec{\nu}_t) \\
        (I + P_t B R^{-1} B^T) \vec{\nu}_t &= P_t A \vec{x}_{t - 1} \\
        \vec{\nu}_t &= {(I + P_t B R^{-1} B^T)}^{-1} P_t A \vec{x}_{t - 1} \\
        A^T \vec{\nu}_t &= A^T {(I + P_t B R^{-1} B^T)}^{-1} P_t A \vec{x}_{t - 1} \\
        &= \vec{\nu}_{t - 1} - Q \vec{x}_{t - 1} \\
        \vec{\nu}_{t - 1} &= (A^T {(I + P_t B R^{-1} B^T)}^{-1} P_t A + Q) \vec{x}_{t - 1}
    \end{align*}
    \tcblower
    Consequently, we obtain the Ricatti Equation:
    \[
        P_{t - 1} = A^T {(I + P_t B R^{-1} B^T)}^{-1} P_t A + Q
    \]
\end{ln-explain}
Therefore, upon our prior work,
\begin{align*}
    \vec{u}_{t - 1}
    &= R^{-1} B \vec{\nu}_t \\
    &= R^{-1} B {(I + P_t B R^{-1} B^T)}^{-1} P_t A \vec{x}_{t - 1} \\
    &= K_{t - 1} \vec{x}_{t - 1} \\
    K_{t - 1} &= R^{-1} B {(I + P_t B R^{-1} B^T)}^{-1} P_t A
\end{align*}
