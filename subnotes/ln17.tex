\chapter{Weak, Strong Duality}

\section{Introduction to Weak vs. Strong Duality via Problems}
\subsection{Minimum-Norm Problem, Continued}
The problem is phrased as:
\[
    \min_{\substack{A \vec{x} = \vec{b} \\ A \vec{x} - \vec{b} = \vec{0}}} \vec{x}^T \vec{x}
\]
where, $A$ is full rank and thus have infinite solutions. \\
We may write the lagrangian of such problem as a convex, quadratic function of $\vec{x}$:
\[
    L(\vec{x}, \vec{\nu}) = \vec{x}^T \vec{x} + {\vec{\nu}}^T (A\vec{x} + \vec{b})
\]
and we may therefore derive that,
\[
    g(\vec{\nu}) = \min_{\vec{x}} L(\vec{x}, \vec{\nu})
\]
Furthermore, as the function $L$ is convex quadratic, we can minimize such function over $\vec{x}$ by setting gradient to be $\vec{0}$:
\begin{align*}
    \nabla_{\vec{x}} L(\vec{x}, \vec{\nu})
    &= 2 \vec{x} + A^T \vec{\nu} \\
    2 \vec{x}^* + A^T \vec{\nu} &= \vec{0} \\
    \vec{x}^* &= -\frac{1}{2} A^T \vec{\nu}
\end{align*}
Therefore,
\begin{align*}
    g(\vec{\nu}) &= L(-\frac{1}{2} A^T \vec{\nu}, \vec{\nu}) \\
    &= \frac{1}{4} \vec{\nu}^T A A^T \vec{\nu} + \vec{\nu}^T (-\frac{1}{2} A A^T \vec{\nu} - \vec{b}) \\
    &= - \frac{1}{4} \vec{\nu}^T A A^T \vec{\nu} - \vec{\nu}^T \vec{b}
\end{align*}
is a concave quadratic function of $\vec{\nu}$.

We also understand that $g(\vec{\nu})$ is a lowerbound to the true minimum norm solution (see note 16). \\
Furthermore, via vector calculus, we may obtain that:
\[
    \underset{\vec{\nu}}{\rm argmax}\ g(\vec{\nu}) = -2 {(A A^T)}^{-1} \vec{b}
\]
And therefore, the true lower lower bound to minimum solution is:
\[
    \vec{x}^* = -\frac{1}{2} A^T \vec{\nu}^* = A^T {(A A^T)}^{-1} \vec{b}
\]
As we have so obtained in the first notes. \\
\begin{ln-define}{Strong vs. Weak Duality}{}
    We phrase that,
    \[
        d^* = \max_{\vec{\nu}} g(\vec{x}^*, \vec{\nu})
    \]
    is a \textbf{dual problem} of the primal, which is the maximization of a concave function $g$.
    In such situation, where $p^*$ is equal to its tightest lowerbound $d^*$, we understand this as \textbf{``strong duality''}. \\
    Otherwise, when $p^* \geq d^*$, we phrase this situation as \textbf{weak duality}.
\end{ln-define}

\subsection{Partitioning Problem}
Problem statement:
\begin{quote}
    For some symmetric matrix $W \in \mathbb{S}^n$, solve:
    \[
        \min_{\substack{\vec{x} \\ x_i^2 = 1}} \vec{x}^T W \vec{x}
    \]
\end{quote}
this is not necessarily a convex optimization problem, and its combinatorial bruteforce solution for finding optimum has an exponential complexity in computation.

Let us first phrase the Lagrangian of this function:
\begin{align*}
    L(\vec{x}, \vec{\nu})
    &= \vec{x}^T W \vec{x} + \sum_i \nu_i (x_i^2 - 1) \\
    &= \vec{x}^T (W + diag(\vec{\nu})) \vec{x} - \sum_i \nu_i
\end{align*}
Then, remind that,
\[
    g(\vec{\nu}) = \underset{\vec{x}}{\rm inf}\ L(\vec{x}, \vec{\nu})
\]
where, if $W + diag(\vec{\nu}) \succcurlyeq 0$, then $L(\vec{x}, \vec{\nu})$ is a convex functino of $\vec{x}$, and because the quadratic term of $g$ must be nonnegative, the minimum value of this function $g$ becomes $-\sum_{i = 1}^n \mu_i$. \\
Otherwise, in the case that $W + diag(\vec{\nu}) \preccurlyeq 0$, the minimum of this function becomes $-\infty$. \\
This phrases the function $g$ as a piecewise function:
\[
    g(\vec{\nu}) = 
    \begin{cases}
        -\sum_{i = 1}^n \mu_i &,\text{ if } Q \succcurlyeq 0 \\
        -\infty &,\text{ otherwise}
    \end{cases}
\]
Consequently,
\begin{align*}
    d^*
    &= \max_{\vec{\nu}} g(\vec{x}^*, \vec{\nu}) \\
    &= \max_{\substack{\vec{\nu} \\ W + diag(\vec{\nu}) \succcurlyeq 0}} - \sum_{i = 1}^n \nu_i
\end{align*}
To maximize the above expression, we need that
\[
    \vec{\nu} = - \lambda_{\min} (W) \vec{1}
\]
Meanwhile, we understand that $W - \lambda_{\min} (W) I \succcurlyeq 0$ by the eigenvalue shift property. \\
Therefore, we may state that,
\[
    p^* \geq d^* = n \lambda_{\min} (W)
\]
and conclude a case of weak duality.

\section{Minmax Inequality}
Let there be a primal problem $p$, and a dual problem $d$, such that: \\
\textbf{Primal.}
\[
    p^* = \min f_0 (\vec{x}) + \sum_{i = 1}^m \mathbbm{1}_{<0} \{f_i(\vec{x})\} + \sum_{j = 1}^n \mathbbm{1}_{0} \{h_j(\vec{x})\}
\]
\textbf{Dual.}
\begin{align*}
    g(\vec{x}^*, \vec{\nu}) = L(\vec{x}, \vec{\lambda}, \vec{\nu})
    d^* = \max_{\substack{ \vec{\lambda} \geq 0 \\ \vec{\nu}}} g(\vec{x}^*, \vec{\nu})
\end{align*}
Then, the Minmax Inequality states:
\begin{ln-theorem}{Minmax Inequality}{}
    For any sets $\mathcal{X}, \mathcal{Y}$, function $f$,
    \[
        \min_{\vec{x} \in \mathcal{X}} \max_{\vec{y} \in \mathcal{Y}} f(\vec{x}, \vec{y})
        \geq
        \max_{\vec{y} \in \mathcal{Y}} \min_{\vec{x} \in \mathcal{X}} f(\vec{x}, \vec{y})
    \]
    Essentially, in a minimax perspective, going second is always more advantageous.
    \tcblower
    \textbf{Proof.}
    Consider some $\vec{x}_0 \in \mathcal{X}$, $\vec{y}_0 \in \mathcal{Y}$, then define
    \begin{align*}
        h(\vec{y}_0) := \min_{\vec{x} \in \mathcal{X}} f(\vec{x}, \vec{y}_0) \\
        g(\vec{x}_0) := \max_{\vec{y} \in \mathcal{Y}} f(\vec{x}_0, \vec{y})
    \end{align*}
    Then, let us perform the following algebraic process:
    \begin{align*}
        h(\vec{y}_0)
        &= \min_{\vec{x} \in \mathcal{X}} f(\vec{x}, \vec{y}_0) \\
        &\leq f(\vec{x}_0, \vec{y}_0) \\
        &\leq \max_{\vec{y} \in \mathcal{Y}} f(\vec{x}_0, \vec{y}) = g(\vec{x}_0)
    \end{align*}
    Then, along that inequality which works for any pair of points $(\vec{x}, \vec{y})$,
    \[
        \max_{\vec{y}_0 \in \mathcal{Y}} h(\vec{y}_0) \leq \min_{\vec{x}_0 \in \mathcal{X}} g(\vec{x}_0)
    \]
    The above translates to exactly the Minmax Inequality.
\end{ln-theorem}

While we consider in an optimization problem that its dual is phrased as:
\[
    d^* = \max_{\vec{\lambda}, \vec{\nu}} g(\vec{\lambda}, \vec{\nu}) = \max_{\vec{\lambda}, \vec{\nu}} \min_{\vec{x}} L(\vec{x}, \vec{\lambda}, \vec{\nu})
\]
we also consider the primal of an optimization problem to phrase,
\begin{align*}
    p^* &= \max_{\vec{\lambda}, \vec{\nu}} L(\vec{x}, \vec{\lambda}, \vec{\nu}) \\
    &= \max_{\vec{\lambda}, \vec{\nu}} (f_o(\vec{x}) + \sum_i \lambda_i f_i(\vec{x}) + \sum_i \nu_i h_i (\vec{x})) \\
    &= \begin{cases}
        \infty &,\text{ upon any constraints $f_i, h_i$ violated} \\
        f_0(\vec{x}) &,\text{ if all constraints are unviolated}
    \end{cases} \\
\end{align*}
So, provided that the constraints are unviolated in the primal problem,
\begin{align*}
    p^* &= \min_{\vec{x}} \max_{\vec{\lambda}, \vec{\nu}} (f_o(\vec{x}) + \sum_i \lambda_i f_i(\vec{x}) + \sum_i \nu_i h_i (\vec{x}))
\end{align*}
Therefore, along the Minmax Inequality, we may observe that:
\[
    p^* \geq d^*
\]
