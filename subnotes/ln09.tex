\chapter{Ridge Regression}

\section{Perturbation Analysis Guides into Ridge Regression}
In the concept of perturbation analysis, we ask that, for some system
\[A \vec{x} = \vec{y}\]
with a square, invertible $A$, how much would $\vec{x}$ change provided some small change $\vec{y} \rightarrow \vec{y} + \vec{\partial y}$?

Then, our solution (cited in Chapter 8, or Lecture 8) is as follows:
\[
    \frac{\pnorm{\vec{\partial x}}{2}}{\pnorm{\vec{x}}{2}} \leq \frac{\sigma_{\max}(A)}{\sigma_{\min}(A)} \frac{\pnorm{\vec{\partial y}}{2}}{\pnorm{\vec{y}}{2}}
\]
From whcich, we discover the characteristic of a matrix called \textbf{condition number}:
\[\frac{\sigma_{\max}(A)}{\sigma_{\min}(A)}\]

In the context of least squares problem, we are provided a more robust normal equation via the use of pseudoinverse:
\[
    (A^\dagger A) \vec{x} = A^T \vec{b}
\]
Then, provided that our least square system is sensitive to noise, we may investigate the condition number of $A^T A$ to observe the system's change in its solution provided the perturbation in system measurements. \\
This is significant in computations. A high condition number makes a system's matrix highly unstable for numerical precisions to persist provided noise in measurements and systems.
Such property is, in fact, demonstrated by the convergence warnings in Jupyter iPython notebooks! \\
But, provided the significance of condition number, we may also discuss ways to alleviate its problems.
To reduce the condition number of some matrix, we may attempt to reduce the ratio of singular values via adding some multiple of identity matrix ($\lambda I$) to it.
This allows us to greatly reduce the condition number, shifting it away from instability and being less prone to variance in training data:
\[A^T A + \lambda I\]

\section{Ridge Regression}
In the least squares problem, we have a formulated optimization problem of:
\[
    \min_{\vec{x}} \pnorm{A \vec{x} - \vec{b}}{2}
\]
to offer insight to the optimization problem to prevent its divergence from true parameter value, is to present a new formulation of the problem:
\[
    \min_{\vec{x}} \pnorm[2]{A \vec{x} - \vec{b}}{2} + \lambda^2 \pnorm[2]{\vec{x}}{2}
\]
where, we regularize the least squares solution by stating that large solutions of $\vec{x}$ must lead to unidealistic increase in the cost (loss) function.
We penalize a large least squares solution $\vec{x}$.

Here, using different norms in the regularizer will provide different properties.
The L1 regularizer has the LASSO property (as outlined in DATA C100), while the L2 regularizer we use now provides a convex function as well as a closed-form solution unlike the L1 effort (once again, as outlined in DATA C100).

Let us first compute the gradient of such loss function:
\begin{align*}
    \nabla_{\vec{x}} \pnorm[2]{A \vec{x} - \vec{b}}{2} + \lambda^2 \pnorm[2]{\vec{x}}{2}
    &= 2A^T (A \vec{x} - \vec{b}) + 2 \lambda^2 I \vec{x} \\
    \vec{x}^* &= {(A^T A + \lambda^2 I)}^{-1} A^T \vec{b}
\end{align*}
The eigenvalues of $A^T A + \lambda^2 I$, meanwhile, would be the eigenvalues of $A^T A$ added the value $\lambda^2$ (by manipulating the definition of eigenvalues and eigenvectors). \\
This is also known as the shift property of eigenvalues:
\begin{ln-theorem}{Shift Property of Eigenvalues}{}
    Let $\vec{v}$ be an eigenvector of $A^T A$. \\
    Then, provided that:
    \[
        A^T A \vec{v} = \mu \vec{v}
    \]
    we may see that,
    \[
        (A^T A + \lambda^2 I) \vec{v} = (\mu + \lambda^2) \vec{v}
    \]
    Thus determine that:
    \begin{quote}
        for any eigenpair of $A^T A$ being $(\mu, \vec{v})$, a corresponding eigenpair exists in $A^T A + \lambda^2 I$ being $(\mu + \lambda^2, \vec{v})$.
    \end{quote}
\end{ln-theorem}
Therefore, small $\lambda$ corresponds to less regularizaiton effort, and vice versa.

\subsection{Development of Ridge Regression}
Now, suppose we have a least square system where $\vec{x}$ has a smaller norm, where perhaps $\lambda^2 I \vec{x} \sim \vec{0}$. \\
Then, adding the information of $\lambda^2 I$ into the least squares problem formulation would not alter the original formulation too much, due to the close-to-zero-ness of ridge regression's current least squares solution. \\
This allows us to use larger $\lambda$ (have greater regularization). We have two ways of incorporating such information into the least squares problem now:
\begin{bindenum}
    \item Vertically concatenate $\lambda I$ below $A$.
    \item Use Ridge Regression's format, which uses a similar idea to preserve the original system as muvh as possible.
\end{bindenum}
In the concatenation idea, our system of least squares problem is reformulated as:
\[
    \begin{bmatrix} A \\ \lambda I \end{bmatrix} \vec{x} = \begin{bmatrix} \vec{b} \\ \lambda I \vec{x} \end{bmatrix}
\]
Let's use both block matrix and normal equation to further explore the idea of concatenating $\lambda I$:
\begin{align*}
    &{\bigg( \begin{bmatrix} A^T & \lambda I \end{bmatrix} \begin{bmatrix} A \\ \lambda I \end{bmatrix} \bigg)}^{-1}
    \begin{bmatrix} A^T & \lambda I \end{bmatrix} \begin{bmatrix} \vec{b} \\ \lambda I \vec{x} \end{bmatrix} \\
    &= (A^T A + \lambda^2 I) (A^T \vec{b} + \lambda^2 I \vec{x}) \\
    &\sim (A^T A + \lambda^2 I) A^T \vec{b}
\end{align*}
and we have therefore demonstrated the similarity between the two ideas of ridge regression.

Once we expand the idea to adding weights for matrices, creating the system:
\[
    \begin{bmatrix} W_1 A \\ W_2 I \end{bmatrix} \vec{x} = \begin{bmatrix} W_1 \vec{b} \\ W_2 \vec{x_0} \end{bmatrix}
\]
we end up with a new optimization problem:
\[
    \min_{\vec{x}} \pnorm[2]{W_1 (A \vec{x} - \vec{b})}{2} + \pnorm[2]{W_2 (\vec{x} - \vec{x_0})}{2}
\]
which we call the \textbf{Tikhonov Regularization} technique. \\
\textit{Note: $\vec{x_0}$ is an arbitrary piece of information.}

\section{Probabilistic Information from Ridge Regression}
Suppose that we instead now have probabilistic information:
\[
    (\vec{x_i}, y_i), y_i = g(x_i) + z_i
\]
where, $z_i \sim \mathcal{N}(0, \sigma_i^2)$. \\
And, suppose we have a linear model,
\[
    y_i = \vec{w}^T \vec{x} + z_i
\]
Then, we may state that,
\[
    f_{z_i} (z_i) = \frac{\exp(-\frac{z_i^2}{2\sigma_i^2})}{\sqrt{2 \pi} \sigma_i}
\]
and we attempt to learn the weights $\vec{w}$ to complete the model. \\
Therefore, with multivariate Gaussian noise $\vec{z}$ and datapoint matrices $X$, we formulate this as a least square system,
\[
    X \vec{w} + \vec{z} = \vec{y}
\]

\subsection{Maximum Likelihood Estimation and Maximum A-Posteriori}
Now, we may attempt to estimate the parameters that makes the observed data most likely. \\
This is related to the technique of Maximum Likelihood Estimation. Let me shamelessly copy a segment of my CS189 notes here to provide a brief explanation:
\begin{ln-explain}{MLE from CS189}{}
    Suppose that we go back to the coin flip example (just like 126 does), where heads appear with a probability $p$ (and otherwise for tails). \\
    Then, statisticians world ask, provided the real data of coin, what value of $p$ (the parameter of coin flip probability distribution) is closest to its true inherent value.

    Let us suppose that the number of heads we obtain is a discrete random variable, $X \sim Binomial(n, p)$:
    \[
        \P(X = x) = \binom{n}{x} p^x {(1 - p)}^{n - x}
    \]
    Then, let us propose that the real data presents $k$ heads, and we would defined the Likelihood function $\mathcal{L}$ as:
    \[
        \mathcal{L}(p) = \P(X = k) = \binom{n}{k} p^k {(1 - p)}^{n - k}
    \]
    where it is a function of distribution parameters.
    Then, Maximum Likelihood Estimation (MLE) is the method of estimating the praameters of a statistical distribution by picking the parameters that maximize $\mathcal{L}$.
    Furthermore, it would be a method of density estimation, where we estimate some probability density function from the provided dataset.
\end{ln-explain}
Here, we would thus solve the optimization problem of:
\[
    {argmax}_{\vec{w_0}} f (\vec{y} | \vec{w} = \vec{w_0})
\]
\begin{ln-explain}{Solving Maximum Likelihood Estimation Phrased Optimization}{}
    Let us begin from the prompt:
    \begin{align*}
        {argmax}_{\vec{w_0}} f (\vec{y} | \vec{w} = \vec{w_0})
        &= {argmax}_{\vec{w_0}} \prod_{i = 1}^n f(Y_i = y_i | \vec{w} = \vec{w_0}) \\
        &= {argmax}_{\vec{w_0}} \prod_{i = 1}^n f(z_i = y_i - \vec{x_i}^T \vec{w} | \vec{w} = \vec{w_0}) \\
        &= {argmax}_{\vec{w_0}} \prod_{i = 1}^n \frac{\exp(-\frac{{(y_i - \vec{x_i}^T)}^2}{2\sigma_i^2})}{\sqrt{2 \pi} \sigma_i} \\
        &= {argmax}_{\vec{w_0}} \frac{1}{{\sqrt{2\pi}}^n} \exp \bigg( \sum_{i = 1}^n \frac{-(y_i - \vec{x_i}^T \vec{w})}{2 \sigma_i^2} \bigg) \prod_{i = 1}^n \frac{1}{\sigma_i} \\
        &= {argmin}_{\vec{w_0}} \sum_{i = 1}^n \frac{-{(y_i - \vec{x_i}^T \vec{w})}^2}{2 \sigma_i^2} \\
        &= {argmin}_{\vec{w_0}} \pnorm[2]{S(X \vec{w_0} - \vec{y})}{2}
    \end{align*}
    where,
    \[
        S =
        \begin{bmatrix}
            \frac{1}{\sqrt{2} \sigma_1} & & \\
            & \ddots & \\
            & & \frac{1}{\sqrt{2} \sigma_n}
        \end{bmatrix}
    \]
    and we end up on a weighted least squares setup. \\
    Furthermore, upon IID uniform Gaussians, we will end up with $S = \frac{1}{\sqrt{2} \sigma_i}$.
\end{ln-explain}

Meanwhile, the Maximum A-Posteriori approach attempts to provide a priori distribution on $\vec{w}$, which is precisely what we attempt with CS189's Bayesian Analysis.
We solve the optimization problem of:
\[
    {argmax}_{\vec{w_0}} f(\vec{w} = \vec{w_0} | \vec{y})
\]

As you may observe, both techniques offer the most likely weights for some condition, but the conditions are framed quite differently:
\begin{bindenum}
    \item MLE find the most likely $\vec{w}$ to \textbf{lead to current observation}.
    \item MAP find the most likely $\vec{w}$ \textbf{given the current observation} that occurs.
\end{bindenum}
