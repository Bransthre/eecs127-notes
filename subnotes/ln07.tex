\chapter{Vector Calculus}
The motivation of vector calculus is to work with linear algebra problems more analytically, in more interpretable methods. \\
The tools that help to optimize linear algebra expression is the calculus of vector: vector calculus.

\section{Function Expansion: Taylor Series}
Let's start with functions, the basic of calculus.
\begin{ln-define}{Scalar Valued Function of Vector}{}
    Let a function $f$ be such that:
    \[
        f(\vec{x}): \R^n \rightarrow \R
    \]
    Such a function is by definition a scalar valued function of vector.
\end{ln-define}
Then, we may begin our effort to generalize the derivative of such functions, as follows.

First of all, let us recap Calculus II knowledge of polynomial expansion of some fucntion:
\begin{ln-theorem}{Taylor's Theorem (Expansion)}{}
    Let $f$ be a scalar valued function for scalar:
    \[
        f(x) : \R \rightarrow \R, x_0 \in \R
    \]
    Then, Taylor expansion allows us to write such that:
    \[
        f(x_0 + \Delta x) = f(x_0) + \dv{f}{x} \bigg|_{x = x_0} \Delta x + \cdots
    \]
    Essentially, 
    \[
        f(x_0 + \Delta x) = \sum_{i = 0}^n \frac{f^{(i)}(x_0) {(\Delta x)}^i}{i!}
    \]
    where we call the Taylor expansion towards the $n^{th}$ degree term be an $n^{th}$ order Taylor expansion.
\end{ln-theorem}
What is the purpose of introducing Taylor's expansion here? The fact is, we may approximate a value $f(x_0 + \Delta x)$ based on $f(x)$ along the product of $\Delta x$ and a derivative.
\textbf{The derivative affects how large the ``perturbation'' is that occurs in the approximation of $f(x_0 + \Delta x)$}.
Similar logic is applied for upcoming $n$-order derivatives.

Then, along the above reintroduction of Taylor's Series, let us look at a vector edition of Taylor's Theorem:
\begin{ln-theorem}{Taylor's Theorem for Vectors}{}
    Let $f$ be a scalar valued function for vector. \\
    Then, we may express that:
    \begin{align*}
        f(\vec{x_0} + \Delta \vec{x})
        &= f(\vec{x_0}) + \pdv{f}{x} \bigg|_{\vec{x} = \vec{x_0}} \Delta \vec{x} + \frac{1}{2!} {(\Delta \vec{x})}^T \nabla^2 f(\vec{x}) \bigg|_{\vec{x} = \vec{x_0}} \Delta \vec{x}
    \end{align*}
\end{ln-theorem}
We usually stop at the second order approximation when using Taylor's Theorem for vectors, because this is usually a good equilibrium point for computational precision and simplification.

\section{Function Expansion: Derivative of Vector Functions}
In addition, we define the derivative of vectors as follows:
\begin{ln-define}{Derivatives of Scalar Valued Vector Function}{}
    Here, we observe that the first order derivative of such functions are row vectors, and the second order derivative is a matrix called \textbf{Hessian}:
    \begin{align*}
        \pdv{f}{x} = {(\nabla_{\vec{x}}f)}^T &= \begin{bmatrix} \pdv{f}{x_1} & \cdots & \pdv{f}{x_n} \end{bmatrix} \\
        {(\nabla^2 f(\vec{x}))}_{i,j} &= \pdv{f}{x_i}{x_j} \\
        \nabla^2 f(\vec{x}) &\in \mathbb{S}^n
    \end{align*}
\end{ln-define}
Note that it is only for convex functions where the symmetric argument of Hessian applies. \\
This is beacuse:
\[
    \pdv{f}{x_i}{x_j} = \pdv{f}{x_i} \pdv{f}{x_j}
\]
where the order of variables put matters for functions. \\
The reason why convex functions can host the symmetric Hessian is because of Clairaut's Theorem (not explicitly mentioned in lecture):
\begin{ln-theorem}{Clairaut's Theorem}{}
    If the second partial derivatives of a function are continuous, then the order of differentiation is immaterial. \\
    Alternatively (quoted from Purdue University),
    \begin{quote}
        Let $f : \R^2 \rightarrow \R$ have all partial derivatives up to second derivative be continuous near $(a, b)$, then:
        \[
            \pdv{f}{x}{y} (a, b) = \pdv{f}{y}{x} (a, b)
        \]
    \end{quote}
\end{ln-theorem}

\subsection{Examples of Polynomial Expansion}
Now, let us provide an example in approximating a scalar valued vector function via Taylor's Theorem:
\begin{ln-explain}{Polynomial Expansion of 2-Norm}{}
    Let us attempt to find a Taylor Expansion for the function
    \[f(\vec{x}) = \pnorm[2]{\vec{x}}{2}\]
    We will define the level sets of this function as:
    \[
        \{\vec{x} \big| f(\vec{x}) = C\}
    \]
    which would be circles centered at the origin with radius $\sqrt{C}$. \\
    The gradient of such function would be:
    \[
        \nabla_{\vec{x}}f = \begin{bmatrix} 2x_1 \\ 2x_2 \end{bmatrix}
    \]
    And the Hessian:
    \[
        H_{\vec{x}}f = \begin{bmatrix} 2 & 0 \\ 0 & 2 \end{bmatrix}
    \]
    Therefore, upon Taylor's Theorem,
    \begin{align*}
        f(\vec{x_0} + \Delta \vec{x})
        &= f(\vec{x_0}) + {(\nabla_{\vec{x}}f)}^T \Delta \vec{x} + \frac{1}{2!} {(\Delta \vec{x})}^T H_{\vec{x}}f \Delta \vec{x} \\
        &= {x_0}_1^2 + {x_0}_2^2 + 2{x_0}_1 \Delta x_1 + 2{x_0}_2 \Delta x_2 + {(\Delta x_1)}^2 + {(\Delta x_2)}^2 \\
        &= {({x_0}_1 + \Delta x_1)}^2 + {({x_0}_2 + \Delta x_2)}^2 = \pnorm[2]{\vec{x_0} + \Delta \vec{x}}{2}
    \end{align*}
\end{ln-explain}
Since the Hessian is symmetric, we can perform a lot of interesting mathematics with it. \\
A third order derivative is known as a tensor, but this is out of scope for EECS 127.

Let's look at some more examples of Taylor Expansion:
\begin{ln-explain}{Polynomial Expansion of Euclidean Inner Product}{}
    Let us find a Taylor Expansion to the function:
    \[
        f(\vec{x}) = \vec{x}^T \vec{a}, \vec{a} \in \R^n
    \]
    The gradient of such function is then,
    \[
        \nabla_{\vec{x}} f = \vec{a}
    \]
    Then, the Hessian of such function would be a zero matrix. \\
    Therefore,
    \begin{align*}
        f(\vec{x_0} + \Delta \vec{x})
        &= f(\vec{x_0}) + {(\nabla_{\vec{x}} f)}^T \Delta \vec{x} \\
        &= \sum_{i = 1}^n a_i ({x_0}_i + \Delta x_i) = {(\vec{x_0} + \Delta \vec{x})}^T \vec{a}
    \end{align*}
\end{ln-explain}


\subsection{Matrix in Vector Calculus}
Let us discuss how the roles of matrices are when we take the derivative of a scalar-valued function by a vector.
\begin{ln-explain}{Derivatives of Matrix-Involving Inner Product}{}
    Let us find the derivative expressions to the function:
    \[
        f(\vec{x}) = \vec{x}^T A \vec{x}, A \in \R^{n \times n}
    \]
    We can derive the gradient as follows:
    \begin{align*}
        \frac{\partial}{\partial x_i} \vec{x}^T A \vec{x}
        &= \frac{\partial}{\partial x_i} \vec{x}^T \begin{bmatrix} \vec{A_1} & \cdots & \vec{A_n} \end{bmatrix} \vec{x} \\
        &= \frac{\partial}{\partial x_i} \begin{bmatrix} \vec{x} \cdot \vec{A_1} & \cdots & \vec{x} \cdot \vec{A_n} \end{bmatrix} \vec{x} \\
        &= \frac{\partial}{\partial x_i} \sum_{k = 1}^n \sum_{j = 1}^n A_{j, k} x_k x_j \\
        &= \frac{\partial}{\partial x_i} \bigg( \sum_{j = 1}^n A_{j, i} x_i x_j + \sum_{k = 1}^n A_{i, k} x_i x_k \bigg) \\
        &= \sum_{j = 1}^n A_{j, i} x_j + \sum_{k = 1}^n A_{i, k} x_k
        = \vec{(A^T)}_i \cdot \vec{x} + \vec{A}_i \cdot\vec{x} \\
        \frac{\partial}{\partial x} \vec{x}^T A \vec{x} &= (A + A^T) \vec{x}
    \end{align*}
    And, uh, Hessians.
    \begin{align*}
        \nabla^2 f(\vec{x})
        &= \frac{\partial}{\partial x} (A + A^T) \vec{x} = A + A^T
    \end{align*}
\end{ln-explain}

Then, let us define the derivative of a vector-valued function with respect to some vector variable:
\begin{ln-define}{Jacobian (Derivative) of Vector-Valued Vector Function}{}
    For some function $f(\vec{x}) = \vec{y}$ such that:
    \[
        f: \R^n \rightarrow \R^m
    \]
    Then,
    \[
        J = \begin{bmatrix} \pdv{\vec{f}}{x_1} & \cdots & \pdv{\vec{f}}{x_n} \end{bmatrix}
    \]
\end{ln-define}

\begin{ln-explain}{The Jacobian of Polar-Cartesian Coordinate Translator}{}
    Let $f$ be the function:
    \[
        f(\vec{v}) = \begin{bmatrix} r \cos(\theta) \\ r \sin(\theta) \end{bmatrix}
    \]
    Then the Jacobian of it would be:
    \[
        J =
        \begin{bmatrix}
            \cos(\theta) & -r \sin(\theta) \\
            \sin(\theta) & r \cos(\theta)
        \end{bmatrix}
    \]
    whose determinant would then be $\det(J) = r$.
\end{ln-explain}

Last but not least, let us explore chain rule in matrix calculus:
\begin{ln-define}{Chain Rule}{}
    Let a function $f$ be
    \[
        f(\vec{x}) = g(h(\vec{x}))
    \]
    Then, we may express that
    \[
        \dv{f}{x} = \dv{g}{y} \dv{h}{x}
    \]
    \tcblower
    \textbf{Example.} \\
    Now, let's look at the least squares problem. \\
    Originally, we perform this computation:
    \begin{align*}
        f(x)
        &= \pnorm[2]{A \vec{x} - \vec{b}}{2} \\
        &= {(A \vec{x} - \vec{b})}^T ((A \vec{x} - \vec{b})) \\
        &= \vec{x}^T A^T A \vec{x} - 2 \vec{x}^T A \vec{b} + \vec{b}^T \vec{b} \\
        \nabla_{\vec{x}} f(\vec{x})
        &= (A^T A + A^T A) \vec{x} - 2 A^T \vec{b} \\
        &= 2 A^T A \vec{x} - 2 A \vec{b} \\
        \vec{x}^* &= {(A^T A)}^{-1} A^T \vec{b}
    \end{align*}
    Instead, we may use chain rule:
    \begin{align*}
        \dv{f}{x}
        &= \dv{g}{y} \dv{h}{x} \\
        &= {(\nabla_{\vec{y}} g)}^T \dv{}{x} {} (A \vec{x}) \\
        &= 2(A \vec{x} - \vec{b})^T \cdot A
    \end{align*}
    We may thus obtain the gradient to be:
    \[
        \nabla_{\vec{x}} f = 2 A^T (A \vec{x} - \vec{b})
    \]
\end{ln-define}
