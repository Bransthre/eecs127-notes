\chapter{Strong Duality and Optimality Conditions}

\section{Strong Duality}
We may observe a duality gap of some optimization problem as $p^* - d^*$; then, when is this duality gap $0$, such that we achieve strong duality?
The cndition of strong duality is phrased as follows:
\begin{ln-define}{Slater's Condition}{}
    Assume that the objective function $f_0 (x)$ is convex, constraints $f_i(x)$ are convex, and $h_i(x)$ are affine. \\
    Then, suppose
    \[
        (\exists \text{ a \textit{feasible} }\vec{x}_0 \in ReInt(D) \land \forall i \in \{1, \dots, m\} (f({\vec{x_0}}) < 0)) \implies (p^* = d^*)
    \]
    where $ReInt(D)$ stands for the ``relative interior'' of the domain of $f$. \\
    Here, \textit{feasible} can also be phrased as fitting another statement existing in the constraint, say $A \vec{x}^* = \vec{b}$; in other words, being in the \textit{feasible} region of an optimization.
\end{ln-define}
We can further refine the above condition:
\begin{ln-define}{Refined Slater's Condition}{}
    Let us have an optimization problem:
    \[
        p^* = \min_{
            \substack{
                \forall i \in \{1, \dots, k\}, f_i(x) \leq 0 \text{ and are affine} \\
                \forall i \in \{k + 1, \dots, m\}, f_l(x) \leq 0 \text{ and are non-affine} \\
                \forall j \in \{1, \dots, n\}, h_j(x) = 0 \text{ and are affine}
            }
        } f_0 (x)
    \]
    Then, if there exists some $\vec{x}_0$ in the relative interior of domain of $f$ and also within the feasible region of optimization such that $\forall l, f_l (\vec{x}) < 0$, we may determine that strong duality holds.
\end{ln-define}

\section{Optimality Conditions}
A certificate helps to pose a upperbound to the difference between the primal and dual feasible point at a current setting $(point, parameter)$.
\begin{ln-define}{Certificate}{}
    Suppose we have a dual feasible point $(\lambda, \nu)$ and primal feasible point $\hat{x}$, then we may suppose that:
    \begin{align*}
        g (\lambda, \nu) \leq d^* \leq p^* \leq f(\hat{x}) \\
        f(\hat{x}) - p \leq f(\hat{x}) - g(\lambda, \nu) \leq \epsilon
    \end{align*}
    Then, we obtain the following conclusion:
    \[
        p^* \in [g(\lambda, \nu), f_0 (\hat{x})]
    \]
    and similarly for $d^*$ upon strong duality.
\end{ln-define}
We, in convex optimization, deal with errors between the primal and dual optimum of some optimization problem. \\
This requires us to bound the error, and implies that our answers might not be the optimal solution. In other words, its optimality is not guaranteed. \\
So how do we find optimality in a solution?

We may attempt to find a defining condition for optimality to occur, where the primal and dual optimal solutions are the same (in the case we simplify a constrained primal optimization problem into an unconstrainted dual optimizartion problem). \\
Suppose our primal optimal point is $\vec{x}^*$ and dual optimal point is $\vec{\lambda}^*, \vec{\nu}^*$, then:
\begin{align*}
    d^* &= \min_{\vec{x}} f_0 (\vec{x}) + \sum \lambda_i^* f_i (\vec{x}) + \sum \nu_j^* h_j (\vec{x}) \\
    &\leq f_0 (\vec{x}^*) + \sum \lambda_i^* f_i (\vec{x}^*) + \sum \nu_j^* h_j (\vec{x}^*) \\
    &\leq \min_{\vec{x}} f_0 (\vec{x}) = p^*
\end{align*}
If the second line of inequality holds, then we observe $\vec{x}^*$ as the minimizer of $L$.
\[
    \nabla_{\vec{x}} L(\vec{x}, \vec{\lambda}, \vec{\nu}) = \vec{0}
\]
Furthermore, suppose the third line of inequality that $\sum_i \lambda_i^* f_i(\vec{x}) = 0$.
Since this is a sum of nonpositive values, if strong duality holds such that $p^* = d^*$, the following condition called \textbf{complementary slackness} us be satisfied:
\[
    \forall i \in \{1, \dots, n\} \lambda_i^* f_i(\vec{x}) = 0
\]
Complementary slackness can be utilized in the case where strong duality holds, and the constraints are differentiable; therefore, this technique is also suited for non-convex problem.

Now, we further extend upon these arguments to form the KKT conditions:
\begin{ln-define}{KKT Condition}{}
    KKT (Karush Kuhn Tucker) Condition is a series of conditions where, assuming strong duality holding, we may observe the following statements being satisfied at an optimal point, along each partition of the conditions: \\
    \textbf{Stationarity}:
    \[
        \forall i \in \{1, \dots, m\}, \nabla f_0(x^*) + \sum \lambda_i^* {f_i(x^*)} + \sum \nu_j^* \nabla h_j (x^*) = 0
    \]
    \textbf{Primal Feasibility}:
    \[
        \begin{cases}
            \forall i \in \{1, \dots, m\}, &f_i(x^*) \leq 0 \\
            \forall j \in \{1, \dots, n\}, &h_j(x^*) = 0
        \end{cases}
    \]
    \textbf{Dual Feasibility}:
    \[
        \forall i \in \{1, \dots, m\}, \lambda_i^* \geq 0
    \]
    \textbf{Complementary Slackness}:
    \[
        \forall i \in \{1, \dots, n\}, \lambda_i^* f_i(\vec{x}) = 0
    \]
\end{ln-define}
If the point is optimal, then KKT condition holds; however, the converse is not necessarily true.
Then, restricting the space of points satisfying KKT is not really a restirction for optimality, but every point that satisfies KKT need not be optimal. \\
However, if our optimization problem is convex, then KKT conditions are sufficient to indicate optimality such that:
\begin{quote}
    If $(\tilde{x}, \tilde{\lambda}, \tilde{\nu})$ satisfies KKT condition, then $\tilde{x}$ is a primal optimum and $(\tilde{\lambda}, \tilde{\nu})$ is a dual optimum.
\end{quote}

\textbf{Summary.} As we ensure the strong duality of a problem by Slater's Condition, we then check for KKT condition to restrict ourselves onto a subset with possibly optimal points; and, in convex optimization problems, KKT conditions guarantee the optimality of its satisfying points.

\section{Linear Program}
\begin{ln-define}{Linear Program}{}
    A linear program is an optimization in the form of:
    \[
        p^* = \min_{
            \substack{
                \forall i \in \{1, \dots, m\}, \vec{a}_i^T \vec{x} \leq b_i \\
                \forall j \in \{1, \dots, n\}, \vec{u}_j^T \vec{x} = v_j
            }
        } (\vec{c}^T \vec{x})
    \]
    where its constraints are all affine.
    Equivalently, we may writea linear program as
    \[
        p^* = \min_{
            \substack{
                A \vec{x} \leq \vec{b} \\
                U \vec{x} = \vec{v}
            }
        } (\vec{c}^T \vec{x})
    \]
\end{ln-define}
Then, the Lagrangian of a linear program can be derived as:
\begin{align*}
    L(\vec{x}, \vec{\lambda}, \vec{\nu}) &= \vec{c}^T \vec{x} + \sum_i \lambda_i (A \vec{x} - \vec{b}) + \sum_i \nu_j (U \vec{x} - \vec{v}) \\
    g(\vec{\lambda}, \vec{\nu}) &= \min_{\vec{x}} L(\vec{x}, \vec{\lambda}, \vec{\nu})
\end{align*}
Then, at some optimal point $\vec{x}^*$, we observe that
\[
    \nabla_{\vec{x}} L(\vec{x}, \vec{\lambda}, \vec{\nu}) = c + A^T \lambda + U^T \nu = 0
\]
Therefore, to derive for the optimal solution, we may find an equivalent optimization problem to work on:
\[
    \max_{
        \substack{
            \forall i \in \{1, \dots, m\}, \lambda \geq 0 \\
            c + A^T \lambda + U^T \nu = 0
        }
    } -(b^T \vec{\lambda} + \vec{\nu}^T \vec{v})
\]
Thus, the dual of a linear program can also be phrased as a linear program.
