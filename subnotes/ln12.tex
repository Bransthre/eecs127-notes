\chapter{Descent Methods}

\section{Strict Strong Convexity}
Convexity also comes in different strictness. These more refined definition provide us flexibility and rigor in proofs. \\
For example, we may first consider that, the zeroth order definition of convexity would be:
\begin{quote}
    For a function $f: \R^n \rightarrow \R$, the domain of $f$ needs to be convex, and Jensen's Inequality applies to any two arbitrary points on the domain.
\end{quote}
However, we fail to classify linear functions as either convex or concave. Linear functions were concluded to be both convex and concave, which is very confusing.
\begin{ln-define}{Strict Convexity}{}
    The zeroth order condition of strict convexity is altered to:
    \begin{quote}
        For a function $f: \R^n \rightarrow \R$, the domain of $f$ needs to be convex, and:
        \[
            f(\theta \vec{x} + (1 - \theta) \vec{y}) < \theta f(\vec{x}) + (1 - \theta) f(\vec{y})
        \]
    \end{quote}
    The first order condition is altered to:
    \[
        \forall \vec{x}, \vec{y} \in Domain(f), f(\vec{y}) > f(\vec{x}) + \dv{f}{x} \cdot (\vec{y} - \vec{x})
    \]
    The second order condition is altered along a very similar logic:
    \[
        \nabla^2 f(\vec{x}) > 0
    \]
    And in summary, a strictly convex function must have a unique minimum (unlike, say, a ReLU function).
\end{ln-define}

But just like humans, not only can convexity have strict variants, they can also have strong variants.
Strong convexity implies strict convexity, and strict convexity implies convexity. These definitions are increasingly difficult to satisfy:
\begin{ln-define}{Strong Convexity}{}
    If $f$ is differentiable, then for some $\mu > 0$, it is $\mu$-strongly convex if domain of $f$ is convex and
    \[
        \forall \vec{x}, \vec{y} \in domain(f), f(\vec{y}) \geq f(\vec{x}) + \dv{f}{x} \cdot (\vec{y} - \vec{x}) + \frac{\mu}{2} \pnorm[2]{\vec{y} - \vec{x}}{2}
    \]
\end{ln-define}
The insight such variant provides is, for some $\mu$-strongly convex function:
\[
    \nabla^2 f(x) \succcurlyeq \mu I \implies \nabla^2 f(x) - \mu I \succcurlyeq 0
\]
The smallest possible eigenvalue of the Hessian of $f$ is then $\mu$.
The larger such value $\mu$ is, the stronger its convex; in other words, the further the function is from being nonconvex.

\section{Gradient Descent}
As you might have learned in prior courseworks (during college, or, high school given Berkeley), there are many variants of Gradient Descent. \\
Let's think about the vanilla version of gradient descent first, as an unconstrained optimization problem:

\subsection{Inventing Gradient Descent}
\[
    p^* = \min_{\vec{x} \in \R^n} f_0(\vec{x})
\]
The foundation of gradient descent algorithm is an ``oracle'' that \underline{guides us at the direction of some greatest descent} along the provided function $f$. \\
Now, we should understand when does the ``oracle'' guide us to convergence, such that we do not get lost in the search space.
\par
Let's find the ``oracle'' via some perspectives on linear perturbation:
\begin{align*}
    f(\vec{x} + \Delta \vec{x}) = f(\vec{x}) + \dv{f}{x} \cdot (\Delta \vec{x}) + \text{H.O.T.}
\end{align*}
If $\Delta \vec{x}$ is a good direction to perturb $\vec{x}$ to, then we wish as well that $f(\vec{x} + \Delta \vec{x}) < f(\vec{x})$. \\
Suppose that $\Delta \vec{x} = s \vec{v}$, where $s$ is a real scalar and $\vec{v}$ is some direction-resembling vector. \\
Then, let us rewrite $f(\vec{x} + s \vec{v})$ as:
\begin{align*}
    f(\vec{x} + s \vec{v})
    &= f(\vec{x}) + s \dv{f}{x} {} \vec{v} \\
    &= f(\vec{x}) + s \langle \nabla f(\vec{x}), \vec{v} \rangle < f(\vec{x}) \\
    \langle \nabla f(\vec{x}), \vec{v} \rangle &< 0
\end{align*}
This means $\nabla f(\vec{x})$ and $\vec{v}$ should be in opposite direction. \\
Therefore,
\[
    \vec{v} = - \nabla f(\vec{x})
\]
and we conclude that the opposite direction of gradient is the great direction of descent-- the ``oracle''.
\par
Now, let us write up a draft for the gradient descent algorithm (which I will follow the minimal-energy principle of Physics and shamelessly copy from my notes in prior courseworks), that:
\begin{ln-define}{Gradient Descent Algorithm}{}
    Let the initial guess of minimum be $\vec{x_0}$. \\
    Then, provided a stepsize $\eta$, and defining $\vec{x_k}$ to be the $k^{th}$ guess of minimum, the iterative approach of algorithm follows as:
    \[
        \vec{x}_{k + 1} = \vec{x}_k - \eta \nabla f(\vec{x_k})
    \]
\end{ln-define}
To get better constant stepsizes, we can perform hyperparamter tuning; or, solve the next subsection.

\subsection{Finding $\eta$}
Let us first define the function we perform gradient descent on, to demonstrate the process of finding $\eta$. \\
Let
\begin{align*}
    f_0 (\vec{x}) &= \pnorm[2]{A \vec{x} - \vec{b}}{2} \\
    \nabla f_0 (\vec{x}) &= 2 A^T (A \vec{x} - \vec{b})
\end{align*}
The iterative rule of gradient descent on function $f_0$ is thus:
\begin{align*}
    \vec{x}_{k + 1}
    &= I \vec{x}_k - 2 \eta A^T (A \vec{x} - \vec{b}) \\
    &= (I - 2 \eta A^T A) \vec{x}_k + 2 \eta A^T \vec{b}
\end{align*}
Let us discuss the recursion in the above form. \\
First of all, we require that the matrix coefficient of guess term: $(I - 2 \eta A^T A)$, to provide some scaling that does not consistently scale the vector larger, so to prevent divergence.
In matrices, we may consider eigenvalue as the indicator of such divergence or convergence. We require that the eigenvalues of $(I - 2 \eta A^T A)$ satisfy some condition such that it does not lead to a consistently larger guess.
\par
Now, let us look at the guess rule:
\begin{align*}
    \vec{x}_{k + 1} - \vec{x}^*
    &= (I - 2 \eta A^T A) \vec{x}_k + 2 \eta A^T \vec{b} - {(A^T A)}^{-1} A^T \vec{b} \\
    &= (I - 2 \eta A^T A) \vec{x}_k + (2 \eta A^T A - I) {(A^T A)}^{-1} A^T \vec{b} \\
    &= (I - 2 \eta A^T A) [\vec{x}_k - {(A^T A)}^{-1} A^T \vec{b}] \\
    &= (I - 2 \eta A^T A) [\vec{x}_k - \vec{x}^*]
\end{align*}
Solve the recursion above,
\begin{align*}
    \vec{x}_{m + n} - \vec{x}^*
    &= (I - 2 \eta A^T A) [\vec{x}_{m + (n - 1)} - \vec{x}^*] \\
    &\vdots \\
    &= {(I - 2 \eta A^T A)}^n [\vec{x}_{m + (n - n)} - \vec{x}^*]
    = {(I - 2 \eta A^T A)}^n [\vec{x}_m - \vec{x}^*]
\end{align*}
Upon the fact that $I - 2 \eta A^T A$ is a symmetric matrix (i.e., it is diagonalizable), we are granted the following statements:
\begin{align*}
    I - 2 \eta A^T A &= U \Lambda U^T \\
    \vec{x}_k - {(A^T A)}^{-1} A^T \vec{b}
    &= {(I - 2 \eta A^T A)}^k [\vec{x}_0 - {(A^T A)}^{-1} A^T \vec{b}] \\
    &= U \Lambda^k U^T [\vec{x}_0 - {(A^T A)}^{-1} A^T \vec{b}]
\end{align*}
Hinting at what a good choice of $\eta$ is such that we may guarantee the subsequent guesses decrease as timestep increase.
Particularly, we choose some $\eta$ such that the eigenvalues of $(I - 2 \eta A^T A)$ are less than $1$ in magnitude.
This allows us to upperbound the scaling effect of a matrix on some vector to not increase the vector's size (in the sense that, our guess does not travel away from the optimum we attempt to find via gradient descent).

\subsection{Generalizations}
For functions that are $\mu$-strongly convex, their gradients do not change too slowly. \\
For functions that are $L$-smooth, such that
\[
    f(\vec{y}) \leq f(\vec{x}) + \nabla f(\vec{x}) \cdot (\vec{y} - \vec{x}) + \frac{L}{2} \pnorm[2]{\vec{y} - \vec{x}}{2}
\]
their gradients do not change too fast.

\begin{ln-theorem}{Lemma on Gradients of L-smooth Functions}{}
    \textbf{Lemma:} For an L-smooth function $f$
    \[
        \pnorm[2]{\nabla f(\vec{x})}{2} \leq 2L (f(\vec{x}) - f(\vec{x}^*))
    \]
    \tcblower
    Using the $L$-smooth property of a function,
    \begin{align*}
        f(\vec{y})
        &\leq f(\vec{x}) + \nabla f(\vec{x}) \cdot (\vec{y} - \vec{x}) + \frac{L}{2} \pnorm[2]{\vec{y} - \vec{x}}{2} \\
        f(\vec{x} - \eta \nabla f(\vec{x}))
        &\leq f(\vec{x}) + \dv{f}{x} {} (-\eta \nabla f(\vec{x})) + \frac{L}{2} \pnorm[2]{\eta \nabla f(\vec{x})}{2}
    \end{align*}
\end{ln-theorem}
