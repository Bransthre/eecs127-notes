\chapter{Convex Optimization Problems}

\section{Convex Functions, Continued}
We have defined last time that
\begin{ln-define}{Convex Functions}{}
    A scalar-valued function $f: \R^n \rightarrow \R$ is convex iff the following qualities are satisfied:
    \begin{bindenum}
        \item The domain of $f$ is a convex set.
        \item The function $f$ obeys the Jensen's Inequality:
        \[
            \forall \vec{x}, \vec{y} \in Domain(f), \theta \in [0, 1], f(\theta \vec{x} + (1 - \theta) \vec{y}) \leq \theta f(\vec{x}) + (1 - \theta) f(\vec{y})
        \]
    \end{bindenum}
    Essentially, the Jensen's Inequality states that:
    \begin{quote}
        Any chord between two points $(x, f(x)), (y, f(y))$ of a convex function would \underline{not be below} the function curve between those two points $(x, f(x)), (y, f(y))$.
    \end{quote}
\end{ln-define}
Concave functions are, on the other hand, the version of convex functions where many theories apply on maximization. \\
For example, while convex functions' critical points are directly their global minimum, concave functions' critical points are directly their global maximum. \\
By studying convex optimization, we equivalently study many aspects of concave optimization.
(Moreover, inverting the sign of some concave function $f$ already make a convex objective function). \\
Interestingly, linear functions are both concave and convex: they obey Jensen's Inequality by having the chord overlap with the objective function curve itself.

Now, let us discuss the properties of functions further:
\begin{ln-define}{Epigraph}{}
    An epigraph of a function $f$ is the set:
    \[
        epi(f) = \{ (\vec{x}, t) \big| \vec{x} \in dom(f), f(\vec{x}) \leq t\}
    \]
    Reviewing the above definition, we discover that an epigraph is the space above the objective function value, the ``volume above curve''.
\end{ln-define}
Such figure is related to the convexity of a function:
\begin{ln-theorem}{Epigraph and Convexity}{}
    \textbf{Theorem.} Function $f$ is convex iff $epi(f)$ is a convex set.
    \tcblower
    \textbf{Proof.} Let us perform proof(s) for such theorem. \\
    \textbf{Direction 1:} $f$ is convex $\implies$ $epi(f)$ is convex. \\
    A set $C$ is convex if:
    \[
        \forall \vec{x}, \vec{y} \in C, L_{\vec{x}, \vec{y}} = \{\theta \vec{x} + (1 - \theta) \vec{y} | \theta \in [0, 1]\} \subseteq C
    \]
    A function $f$ is convex if:
    \[
        \forall \vec{x}, \vec{y} \in Domain(f), \theta \in [0, 1], f(\theta \vec{x} + (1 - \theta) \vec{y}) \leq \theta f(\vec{x}) + (1 - \theta) f(\vec{y})
    \]
    Let us take two arbitrary points from $epi(f)$ be $(\vec{x}, t_x), (\vec{y}, t_y)$ such that $t_x \geq f(\vec{x})$ and $t_y \geq f(\vec{y})$. \\
    Let us consider some point in $L_{\vec{x}, \vec{y}}$ and see if it belongs to $C$:
    \begin{align*}
        \theta (\vec{x}, t_x) + (1 - \theta) (\vec{y}, t_y)
        &= (\theta \vec{x} + (1 - \theta) \vec{y}, \theta t_x + (1 - \theta) t_y)
    \end{align*}
    Where, via the property of a convex function, we know that $\theta \vec{x} + (1 - \theta) \vec{y}$ is included in its convex domain. \\
    Furthermore, via Jensen's Inequality
    \begin{align*}
        \theta t_x + (1 - \theta) t_y
        &\geq \theta f(\vec{x}) + (1 - \theta) f(\vec{y}) \\
        &\geq f(\theta \vec{x} + (1 - \theta) \vec{y})
    \end{align*}
    Therefore, by definition, a convex function $f$ has a convex epigraph $epi(f)$.
    \par
    \textbf{Direction 2:} $epi(f)$ is convex $\implies$ $f$ is convex. \\
    Take two arbitrary points in $epi(f)$ to be $(\vec{x}, t_x = f(\vec{x})), (\vec{y}, t_y = f(\vec{y}))$, where by the definition of convexity, we realize that:
    \[
        \forall \theta \in [0, 1], (\theta \vec{x} + (1 - \theta) \vec{y}, \theta t_x + (1 - \theta) t_y) \in epi(f)
    \]
    Since such point exists in an epigraph, it must infer that,
    \[
        \theta t_x + (1 - \theta) t_y = \theta f(\vec{x}) + (1 - \theta) f(\vec{y}) \geq f(\theta \vec{x} + (1 - \theta) \vec{y})
    \]
    which is the Jensen's Inequality: what characterizes the definition of a convex function $f$.
\end{ln-theorem}
This thus connects the definitions of convex sets and convex functions.
But, ready or not, there are more definitions of convexity:
\begin{ln-define}{First Order Condition of Convexity}{}
    Let $f$ be a differentiable function, then $f$ is convex iff the following condition is satisfied:
    \[
        \forall \vec{x}, \vec{y} \in Domain(f), f(\vec{y}) \geq f(\vec{x}) + \dv{f}{x} \big|_{\vec{x}} \cdot (\vec{y} - \vec{x})
    \]
    The implication of such definition is:
    \[
        \nabla f(\vec{x}) = 0 \implies \forall \vec{y}, f(\vec{y}) \geq f(\vec{x}) + 0
    \]
    \tcblower
    \textbf{Proof.} \\
    \textbf{Direction 1:} $f$ is convex implies the first order condition of convexity \\
    The definition of convex function allows us to state for some $\theta \in [0, 1]$ that
    \[
        \forall \vec{x}, \vec{y} \in Domain(f), \theta \in [0, 1], f((1 - \theta) \vec{x} + \theta \vec{y}) \leq (1 - \theta) f(\vec{x}) + \theta f(\vec{y})
    \]
    Rearranging the above inequality grants
    \begin{align*}
        \theta f(\vec{y}) &\geq f((1 - \theta) \vec{x} + \theta \vec{y}) - f(\vec{x}) + \theta f(\vec{x}) \\
        f(\vec{y}) &\geq \frac{1}{\theta} f(\vec{x} + \theta (\vec{y} - \vec{x})) - \frac{1}{\theta} f(\vec{x}) + f(\vec{x})
    \end{align*}
    Upon aligning the above equation with the definition of derivative, we may discover that:
    \[
        \lim_{t \rightarrow 0} \frac{f(\vec{x} + \theta (\vec{y} - \vec{x})) - f(\vec{x})}{\theta (\vec{y} - \vec{x})} = \dv{f}{x}
    \]
    Therefore, let us substitute the above derivative term into the function:
    \[
        f(\vec{y}) \geq \dv{f}{x} \cdot (\vec{y} - \vec{x}) + f(\vec{x})
    \]
    which is exactly the first order condition of convexity.
    \par
    \textbf{Direction 2:} the first order condition of convexity implies $f$ is convex. \\
    Suppose we have two arbitrary points $\vec{x}, \vec{y}$, and that:
    \[
        \vec{z} = \theta \vec{x} + (1 - \theta) \vec{y}
    \]
    Let us use the first order condition of convexity to state that,
    \begin{align*}
        f(\vec{x}) &\geq f(\vec{z}) + f'(\vec{z}) (\vec{x} - \vec{z}) \\
        f(\vec{y}) &\geq f(\vec{z}) + f'(\vec{z}) (\vec{y} - \vec{z})
    \end{align*}
    Then,
    \begin{align*}
        \theta f(\vec{x}) + (1 - \theta) f(\vec{y})
        &\geq f(\vec{z}) + \theta f'(\vec{z}) (\vec{x} - \vec{z}) + (1 - \theta) f'(\vec{z}) (\vec{y} - \vec{z}) \\
        &= f(\vec{z}) + \theta f'(\vec{z}) (\theta \vec{x} - \theta \vec{z} + (1 - \theta) \vec{y} - (1 - \theta) \vec{z}) \\
        &= f(\vec{z})
    \end{align*}
\end{ln-define}

Upon the first order condition, we also have a second order condition of convexity:
\begin{ln-define}{Second Order Condition of Convexity}{}
    Let $f$ be a twice-differentiable function, then $f$ is convex iff the following condition is satisfied:
    \[
        \nabla^2 f(x) \succcurlyeq 0
    \]
\end{ln-define}

\section{Convex Optimization Problem}
Let us have some problem:
\[
    p^* = \min f_0(\vec{x}) \text{ s.t. } f_i(\vec{x}) \leq 0
\]
A problem is a convex optimization problem if for all $i$, $f_i(x)$ is a convex function.
