\chapter{Descent Methods and Convex Optimizations}

\section{Continued, Gradient Descent Convergence Proof}
While there was a splendid convergence rate for the least squares problem, we want to discuss how gradient descent operates for other functions. \\
To have some similar quadratic form for the gradient descent convergence proof, we may work with smooth and strongly convex function, and it would offer the interval at which a gradient descent convergence lies at.
In lecture, this was phrased as a ``quadratic sandwich''.
\begin{quote}
    *Puts two bread across the two side of a strongly convex, smooth function \\
    \textbf{WHAT ARE YOU} \\
    \textit{A quadratic sandwich, chef}
\end{quote}
sorry for the digression.

\subsection{Breads of Quadratic Sandwich}
Here, let us review two definitions introduced in the prior lecture:
\begin{ln-define}{Review: Strong Convexity}{}
    If $f$ is differentiable, then for some $\mu > 0$, it is $\mu$-strongly convex if domain of $f$ is convex and
    \[
        \forall \vec{x}, \vec{y} \in domain(f), f(\vec{y}) \geq f(\vec{x}) + \dv{f}{x} \cdot (\vec{y} - \vec{x}) + \frac{\mu}{2} \pnorm[2]{\vec{y} - \vec{x}}{2}
    \]
\end{ln-define}
\begin{ln-define}{L-smooth Functions}{}
    For an $L$-smooth function,
    \begin{align*}
        f(\vec{y})
        &\leq f(\vec{x}) + \nabla f(\vec{x}) \cdot (\vec{y} - \vec{x}) + \frac{L}{2} \pnorm[2]{\vec{y} - \vec{x}}{2} \\
        f(\vec{x} - \eta \nabla f(\vec{x}))
        &\leq f(\vec{x}) + \dv{f}{x} {} (-\eta \nabla f(\vec{x})) + \frac{L}{2} \pnorm[2]{\eta \nabla f(\vec{x})}{2}
    \end{align*}
\end{ln-define}
The L-smoothness of a function guarantees the gradient doesn't change too fast, while the $\mu$-strong convexity guarantees the gradient doesn't change too slow.

Let us prove the lemma addressed in prior lecture regarding the $L$-smooth functions:
\begin{ln-theorem}{Lemma on Gradients of $L$-smooth Functions}{}
    \textbf{Lemma:} For an $L$-smooth function $f$
    \[
        \pnorm[2]{\nabla f(\vec{x})}{2} \leq 2L (f(\vec{x}) - f(\vec{x}^*))
    \]
    \tcblower
    \textbf{Proof:} Suppose the current guess is some vector $\vec{x}$, such that the next guess is $\vec{y} =  \vec{x} - \eta \nabla f(\vec{x})$. \\
    Provided that $f(\vec{x}^*)$ is a global minimum, we may determine that,
    \[
        \forall \vec{x}, f(\vec{x}^*) \leq f(\vec{y}) = f(\vec{x} - \eta \nabla f(\vec{x}))
    \]
    Meanwhile, we may find via the L-smoothness condition that,
    \begin{align*}
        f(\vec{x} - \eta \nabla f(\vec{x}))
        &\leq f(\vec{x}) + \dv{f}{x} {} (-\eta \nabla f(\vec{x})) + \frac{L}{2} \pnorm[2]{\eta \nabla f(\vec{x})}{2} \\
        &= f(\vec{x}) - \eta \pnorm[2]{\nabla f(\vec{x})}{2} + \frac{\eta^2 L}{2} \pnorm[2]{\nabla f(\vec{x})}{2} \\
        &= f(\vec{x}) + \eta \bigg( \frac{\eta L}{2} - 1 \bigg) \pnorm[2]{\nabla f(\vec{x})}{2}
    \end{align*}
    To produce the most efficient upperbound, we should minimize the coefficient of gradient's squared L2 norm. \\
    This brings us to a convex optimization problem:
    \[
        \min_\eta \frac{\eta^2 L}{2} - \eta
    \]
    which we may solve via calculus to see,
    \[
        \eta^* = \frac{1}{L}
    \]
    Let us substitute this back into the above derivation,
    \begin{align*}
        f(\vec{x} - \eta \nabla f(\vec{x}))
        &= f(\vec{x}) + \eta \bigg( \frac{\eta L}{2} - 1 \bigg) \pnorm[2]{\nabla f(\vec{x})}{2} \\
        &= f(\vec{x}) - \frac{L}{2} \pnorm[2]{\nabla f(\vec{x})}{2}
    \end{align*}
    Once again, via the property of $f(\vec{x}^*)$ to be the global minimum,
    \begin{align*}
        f(\vec{x}^*)
        &\leq f(\vec{x} - \eta \nabla f(\vec{x})) \\
        &= f(\vec{x}) - \frac{L}{2} \pnorm[2]{\nabla f(\vec{x})}{2}
    \end{align*}
    This can then be derived into the aforementioned lemma.
\end{ln-theorem}

\subsection{Quadratic Sandwich}
Let us now demonstrate the ``quadratic sandwich'' we discussed in prior:
\begin{ln-theorem}{Quadratic Sandwich}{}
    \textbf{Theorem.} For function $f$ that is $L$-smooth and $\mu$-strongly convex, then for some appropriate $\alpha$,
    \[
        \pnorm[2]{\vec{x}_{t + 1} - \vec{x}^*}{2} \leq \alpha \pnorm[2]{\vec{x}_t - \vec{x}^*}{2}
    \]
    \tcblower
    \textbf{Proof.} For a $\mu$-strongly convex function, we may see that,
    \[
        f(\vec{x}') \geq f(\vec{x}) + \dv{f}{x} \cdot (\vec{x}' - \vec{x}) + \frac{\mu}{2} \pnorm[2]{\nabla f(\vec{x})}{2}
    \]
    Be a little bit manipulative, and we will obtain:
    \[
        \dv{f}{x} \cdot (\vec{x}' - \vec{x}) \leq f(\vec{x}') - f(\vec{x}) - \frac{\mu}{2} \pnorm[2]{\nabla f(\vec{x})}{2}
    \]
    Now, to find a relationship (such that we may continue a proof), we will have to dedicate the following algebraic effort.\\
    Particularly, we attempt to find a relationship between different guesses that are one timestamp across:
    \begin{align*}
        \pnorm[2]{\vec{x}_{t + 1} - \vec{x}^*}{2}
        &= \pnorm[2]{\vec{x}_t - \eta \nabla f(\vec{x}_t) - \vec{x}^*}{2} \\
        &= \pnorm[2]{(\vec{x}_t - \vec{x}^*) - \eta \nabla f(\vec{x}_t)}{2} \\
        &= {\big( (\vec{x}_t - \vec{x}^*) - \eta \nabla f(\vec{x}_t) \big)}^T \big( (\vec{x}_t - \vec{x}^*) - \eta \nabla f(\vec{x}_t) \big) \\
        &= \pnorm[2]{\vec{x}_t - \vec{x}^*}{2} + \pnorm[2]{\eta \nabla f(\vec{x}_t)}{2} + 2 {(\eta \nabla f(\vec{x}_t))}^T {(\vec{x}^* - \vec{x}_t)} \\
        &\leq \pnorm[2]{\vec{x}_t - \vec{x}^*}{2} + 2L \eta^2 (f(\vec{x}_t) - f(\vec{x}^*)) + 2 \eta (f(\vec{x}^*) - f(\vec{x}_t) - \frac{\mu}{2} \pnorm[2]{\nabla f(\vec{x})}{2}) \\
        &= (1 - \eta \mu) \pnorm[2]{\vec{x}_t - \vec{x}^*}{2} + (2L \eta^2 - 2\eta) (f(\vec{x}_t) - f(\vec{x}^*))
    \end{align*}
    Here, by letting $\eta = \frac{1}{L}$, we can cancel out the latter term. \\
    We thus obtain that the appropriate $\alpha$ is:
    \[
        \alpha = 1 - \eta \mu = 1 - \frac{\mu}{L}
    \]
    Therefore, our interval of convergence for $L$ is:
    \begin{align*}
        -1 < 1 - \frac{\mu}{L} < 1 \\
        0 < \mu < 2L
    \end{align*}
\end{ln-theorem}
The implication of quadratic sandwich is that it bounds any function's convergence in gradient descent via some working quadratic forms. \\
If boundable by quadratics, convergence can nicely occur.

\section{Stochastic Gradient Descent}
There is a variety of other names, but SGD is perhaps the most popular choice. \\
Essentially, SGD is ``lazy gradient descent''.
This name comes from its algorithmic property of reducing computational cost to avoid computing an entire gradient for some large vector or dataset.

The loss function of a dataset is very frequently written in the form:
\[
    L(\vec{x}) = \frac{1}{m} \sum_{i = 1}^m L_i (\vec{x})
\]
For example, in least squares algorithm,
\[
    L(\vec{x}) = \frac{1}{m} \pnorm[2]{A \vec{x} - \vec{b}}{2} = \frac{1}{m} \sum_{i = 1}^m {(\vec{a}_i^T \vec{x} - \vec{b_i})}^2
\]
Fortunately, by the additive property of derivatives, we may also state that:
\[
    \nabla f(\vec{x}) = \frac{1}{m} \sum_{i = 1}^m \nabla f_i(\vec{x})
\]
In SGD, instead of stepping in the direction of $\nabla f(\vec{x})$, we take a step in $\nabla f_i (\vec{x})$. \\
This is a legitimate approach, because
\[
    \E[\nabla f_i(\vec{x})] = \frac{1}{m} \sum_{i = 1}^m \nabla f_i (\vec{x}) = \nabla f(\vec{x})
\]

The features of SGD involve:
\begin{bindenum}
    \item Computationally efficient!
    \item Can be comfortably applied on online data
    \item Is a more noisy algorithm, and will thus help to escape local minimum
\end{bindenum}

